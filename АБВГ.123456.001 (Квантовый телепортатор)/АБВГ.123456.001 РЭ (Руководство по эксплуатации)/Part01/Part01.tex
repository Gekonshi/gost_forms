\documentclass[14pt]{extreport}
\usepackage{fontspec}
\usepackage[a4paper,left=25mm,right=10mm,top=15mm,bottom=30mm,footskip=15mm,headsep=5.03978mm]{geometry} % headsep=7.5mm-7pt
\usepackage[colorlinks=true,linkcolor=black,citecolor=black,pdftitle={Квантовый телепортатор. Руководство по эксплуатации},pdfsubject={},pdfkeywords={}]{hyperref}
\usepackage{pdfpages}

\usepackage{gostforms}

\renewcommand{\headrulewidth}{0.0mm}

\pagestyle{fancy}
\fancyhf{}
\fancypagestyle{plain}{\fancyhf{}}

\setcounter{secnumdepth}{3}
\setcounter{tocdepth}{3}

\ifdefined\overlaypass
\newcounter{overlaypage}
\setcounter{overlaypage}{1}
\AddToShipoutPicture{\AtPageLowerLeft{\includegraphics[page=\arabic{overlaypage}]{./Form/Form.pdf}\stepcounter{overlaypage}}}
\fi

\begin{document}
\begin{sloppypar}

% Начало титульного листа
\enlargegosttitle
\gosttitle
% TOP
{\strut}
% TOP LEFT
{
УТВЕРЖДЕНО\\
АБВГ.123456.001-ЛУ
}
% TOP RIGHT
{\strut}
% MIDDLE
{
\uppercase{Квантовый телепортатор}\\*[14pt]
\textbf{Руководство по эксплуатации}\\*[14pt]
\textbf{АБВГ.123456.001 РЭ}
}
% BOTTOM LEFT
{\strut}
% BOTTOM RIGHT
{\strut}
% BOTTOM
{\strut}
% Конец титульного листа

\setcounter{page}{2}
\section*{Аннотация}

Настоящее руководство по эксплуатации квантового телепортатора предназначено для ознакомления пользователей и обслуживающего персонала, содержит указания и рекомендации, необходимые для использования квантового телепортатора по назначению, а так же сведения, необходимые для правильного транспортирования, хранения и обслуживания.

В руководстве приведено техническое описание квантового телепортатора, и представлена информация о составе, технических характеристиках и принципе работы.

\renewcommand\contentsname{\vspace{-32mm}\centerline{\Large{Содержание}}}
\tableofcontents
\newpage

\section{Описание и работа квантового телепортатора}
\subsection{Назначение квантового телепортатора}

На рисунках \ref{fig:vector} и \ref{fig:bitmap} изображено неизвестно что...

\gostfigure{./Pics/vector.pdf}{Векторное изображение}{fig:vector}{0.5}{ht}
\gostfigure{./Pics/bitmap.png}{Растровое изображение}{fig:bitmap}{0.25}{ht}

\clearpage
\section{Эксплуатационные ограничения}

По способу защиты человека от поражения электрическим током устройство относится к изделиям, не представляющим опасность поражения электрическим током и не требующим защиты персонала от случайного соприкосновения с электрическими цепями.

\end{sloppypar}
\end{document}
